%%% Packages
\usepackage{amsmath,amssymb,amsthm}  % extended mathematics
\usepackage{IEEEtrantools}
\usepackage{booktabs}
\usepackage{bm}
\usepackage{dcolumn}
\usepackage{diagbox}
\usepackage{fancyvrb} % extended verbatim environments
\usepackage{float}
\usepackage[colorlinks=true,allcolors=black]{hyperref}
\usepackage{lipsum}   % filler text
\usepackage{microtype}
\usepackage{multirow}
\usepackage[italicdiff]{physics}
\usepackage{siunitx}
\usepackage{textgreek}
\usepackage{tikz}
\usepackage{xfrac}

%%% Mathy Macros
\providecommand{\into}{\hookrightarrow}
\providecommand{\onto}{\twoheadrightarrow}
\providecommand{\ol}{\overline}
\providecommand{\ul}{\underline}
\providecommand{\wt}{\widetilde}
\providecommand{\wh}{\widehat}
\providecommand{\eps}{\varepsilon}
\providecommand{\CC}{\mathbb C}
\providecommand{\FF}{\mathbb F}
\providecommand{\NN}{\mathbb N}
\providecommand{\QQ}{\mathbb Q}
\providecommand{\RR}{\mathbb R}
\providecommand{\ZZ}{\mathbb Z}
\providecommand{\ts}{\textsuperscript}
\providecommand{\dg}{^\circ}
\providecommand{\defeq}{\coloneqq}

\providecommand{\M}{\mathsf{M}}
\providecommand{\U}{\mathsf{U}}
\renewcommand{\O}{\mathsf{O}}
\providecommand{\GL}{\mathsf{GL}}
\providecommand{\SL}{\mathsf{SL}}
\providecommand{\SU}{\mathsf{SU}}
\providecommand{\SO}{\mathsf{SO}}
\providecommand{\Sp}{\mathsf{Sp}}

\providecommand{\fg}{\mathfrak g}
\providecommand{\fh}{\mathfrak h}
\providecommand{\fu}{\mathfrak u}
\providecommand{\fgl}{\mathfrak{gl}}
\providecommand{\fsl}{\mathfrak{sl}}
\providecommand{\fsu}{\mathfrak{su}}

\providecommand{\spec}{\text{spec}}
\providecommand{\Hom}{\text{Hom}}

\DeclareMathOperator*{\lcm}{lcm}
\DeclareMathOperator*{\argmin}{arg min}
\DeclareMathOperator*{\argmax}{arg max}
\DeclareMathOperator*{\sign}{sign}
\DeclareMathOperator*{\diag}{diag}
\DeclareMathOperator*{\spn}{span}

% \renewcommand{\sqrt}[1]{#1^{1/2}}
% \newcommand{\sqrtp}[1]{\sqrt{\qty(#1)}}
% \newcommand{\sqrtb}[1]{\sqrt{\qty[#1]}}

\providecommand{\half}{\frac{1}{2}}
\providecommand{\inv}{^{-1}}

\newcommand{\floor}[1]{\left\lfloor #1 \right\rfloor}
\newcommand{\ceiling}[1]{\left\lceil #1 \right\rceil}
\newcommand{\inner}[2]{\langle #1, #2 \rangle}

\providecommand{\nn}{\nonumber}

% New column types
\newcolumntype{d}[1]{D{.}{.}{#1}}
